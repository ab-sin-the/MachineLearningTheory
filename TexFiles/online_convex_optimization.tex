\documentclass[../main.tex]{subfiles}

\begin{document}
\chapter{Online Convex Optimization}
In this chapter, we will talk about online convex optimization. We will focus on the definition of this problem, some algorithms to solve it and their complexities.

To begin with, let's talk about the framework used in online convex optimization.

\section{Framework}

\begin{itemize}
	\item Given $\K \subset \R^d$, and $\K$ is bounded, convex and closed.
	\item At each round t, learner selects $x_t \in \K$.
	\item Nature reveals a convex function $f_t: \K \to \R$.
	\item Learner wants to minimize $Regret_T = \sum\limits_{t = 1}^T f_t(x_t) - \min\limits_{x\in \K}  \sum\limits_{t = 1}^T f_t(x)$.
\end{itemize}

Let's see three examples about online convex optimization.

\begin{example}[\bfseries Prediction with respect to expert advice]
	The prediction setting we talked about in online learning algorithm is an example of online convex optimization, where
	
	\begin{itemize}
		\item $\K = \Delta_N$.
		\item $f_t(\tilde{w}) = l(\sum\limits_{i=1}^N \tilde{w_i} x_i^t, y^t)$.
	\end{itemize}
\end{example}

\begin{example}[\bfseries Online Portfolio Selection]
	In this setting, we have
	
	\begin{itemize}
		\item $\K = \Delta_N$.
		\item $w^t$ is the distribution of $n$ stocks fluctuations.
		\item $r^t$ is the vector of price, where $r^t(i) = \frac{Price_t(stock_i)}{Price_{t-1}(stock_i)}$.
		\item $f_t(w^t) = -\log(w^t r^t) $.
		\item $\sum\limits_{t=1}^T f_t(w^t) = -\log(\Pi_{t= 1}^T (w^t r^t)) $.
		\item $\min\limits_{w\in \Delta_n} \sum\limits_{t=1}^T f_t(w) = -\log(\max\limits_{w\in \Delta_n} \Pi_{t= 1}^T (w r^t)) $ = negative log wealth of best portfolio.
	\end{itemize}
	
	In this example, basically the loss function is minimized when money is put into the stock whose value will grow higher at time $t$. 
	
	In addition, there exists an algorithm such that for any sequence $r^1, \cdots , r^t$, we can have that 
	
	\begin{equation*}
		\sum\limits_{t=1}^T f_t(w^t) - \min\limits_{w \in \Delta_n} \sum\limits_{t=1}^T f_t(w) \leq n \log T.
	\end{equation*}
\end{example}

\begin{example}[\bfseries Online Regression]
	In this example, basically 
	\begin{itemize}
		\item $\K$ = $2$-norm ball $\Theta$.
		\item $f_t(\theta) = (\theta x_t - y_t)^2$, $(x_t, y_t) \in (\R^n \times \R)$.
	\end{itemize}
\end{example}

Online convex optimization is not an easy problem, it combines ideas from

\begin{itemize}
	\item Optimization
	\item Statistical Learning
\end{itemize}

We have the following claim for (OCO) online convex optimization:

\begin{claim}
	OCO is harder than vanilla optimization.
\end{claim}

\begin{proof}
	The goal of vanilla is to find the solution to $\arg\min\limits_{x\in \K} \Phi(x)$, where $\Phi(\cdot)$ is a convex function.
	
	To prove this claim, let's take any OCO algorithm $\alg$, and take the following procedures.
	
\begin{algorithm}
	\floatname{algorithm}{}
	\begin{algorithmic}
		\STATE Initialize $x_1 \in \K$
		\STATE For $t = 1,\cdots ,T$:
		\bindent 
			\STATE $x_t = \alg(f_1, \cdots, f_{t-1})$
			\STATE Let $f_t(\cdot) = \Phi(\cdot)$ [or = $\langle \nabla \Phi(x_t), \cdot \rangle$]
		\eindent
		\STATE Output $\frac{1}{T} \sum\limits_{t=1}^T x_t = \bar{x_T}$.
	\end{algorithmic}
\end{algorithm}

By this algorithm, we can show that $\Phi(\bar{x_T}) - \min\limits_{x \in \K} \Phi(x) \leq \frac{Regret_T(\alg)}{T}$, since

\begin{equation*}
	\begin{aligned}
		\Phi(\bar{x_T}) & \leq \frac{1}{T} \sum\limits_{t=1}^T \Phi(x_t) \\
		& = \frac{1}{T}  \sum\limits_{t=1}^T f_t(x_t) \\
		& \leq \frac{1}{T}\min\limits_{x \in \K}  \sum\limits_{t=1}^T  f_t(x_t) +  \frac{Regret_T(\alg)}{T} \\
		& \leq \frac{1}{T}\min\limits_{x \in \K}  \sum\limits_{t=1}^T  \Phi(x_t) +  \frac{Regret_T(\alg)}{T} \\
		& =  \min\limits_{x \in \K}  \Phi(x_t) +  \frac{Regret_T(\alg)}{T}.
	\end{aligned}
\end{equation*}
\end{proof}

Therefore, if we can solve online convex optimization problem, then we can also solve vanilla optimization.

Next, we will talk about Online-to-Batch Conversion, which use online optimization to solve batch minimization. The setting is following:

\begin{itemize}
	\item $X$ - Data Space.
	\item $Y$ - Label Space.
	\item $D$ - Distribution over $X \times Y$.
	\item $\Theta \in \R^d$ - Bounded parameter space.
	\item $l: \Theta \times X \times Y \to \R$ - Loss function and convex in $\Theta$.
\end{itemize}

We can pick any OCO algorithm $\alg$ and use it to generalize to batch minimization as following:

\begin{algorithm}
	\floatname{algorithm}{}
	\begin{algorithmic}
		\STATE Initialize $\theta_1 \in \Theta$
		\STATE For $t = 1,\cdots ,T$:
		\bindent 
		\STATE $\theta_t = \alg(f_1, \cdots, f_{t-1})$
		\STATE Let $f_t(\cdot) = l(\cdot; (x_t, y_t))$
		\eindent
		\STATE Output $ \bar{\theta_t} =  \frac{1}{T} \sum\limits_{t=1}^T \theta_t$.
	\end{algorithmic}
\end{algorithm}

By this generalization, we can have that

\begin{theorem}
	Let risk $R(\theta) = \EE_{(x,y) \in D} [l(\theta;(x,y))]$, then we can have
	\begin{equation}
		\EE[R(\bar{\theta_T})] - \min\limits_{\theta \in \Theta} R(\theta) \leq \EE[\frac{Regret_T(\alg)}{T}].
	\end{equation}
\end{theorem}

Afterwards, we will introduce several algorithms in online gradient descent. The first one is called online gradient descent.

\section{Online Convex Optimization Algorithm}
\begin{algorithm}[H]
	\caption{Online Gradient Descent}
	\begin{algorithmic}
		\STATE Initialize $x_1 \in \K$
		\STATE For $t = 1,\cdots ,T$:
		\bindent 
		\STATE $x_{t+1} = \Pi_{\K} (x_t - \eta_t\nabla f_t(x_t))$.
		\eindent
	\end{algorithmic}
\end{algorithm}


where $\Pi_{\K}(\cdot)$ is defined as the projection operator and $\Pi_{\K}(y) = \arg\min\limits_{x\in \K} ||y-x||_2$.

This is a very simple algorithm similar to projected gradient algorithm, except that at each step, we use $f_t$ to calculate the gradient rather than a fixed $f$.

The following convergence theorem is associated with online gradient descent:

\begin{theorem}
	Assume that $f_t$ is $G$-Lipschitz ($||\nabla f_t||\leq G$), and $\forall x, y\ \in D$, $||x - y||\leq D$. Then, with $\eta_t = \frac{D}{G\sqrt{t}}$, we can get that 
	
	\begin{equation}
		Regret_T(OGD) \leq \frac{3}{2} GD\sqrt{T}.
	\end{equation}
\end{theorem}

Before proving this theorem, we will first introduce a lemma about the projection operator.

\begin{lemma}[\bfseries Pythagorean Theorem for Bregman Divergences]\label{pythagorean_lemma}
	For any $x \in \K$, 
	
	\begin{equation*}
	 	||x - \Pi_k(y)|| \leq ||x - y||.
	\end{equation*}
\end{lemma}

Then, let's start proving the convergence of OGD.

\begin{proof}
	Let $x^* = \arg\min\limits_{x\in \K} \sum\limits_{t=1}^T f_t(x)$, and $\nabla_t = \nabla f_t(x_t)$.
	
	By the convexity of $f_t(\cdot)$, we know that 
	
	\begin{equation*}
		f_t(x_t) - f_t(x^*) \leq \nabla_t^T (x_t - x^*).
	\end{equation*}
	
	By lemma \ref{pythagorean_lemma} and the update function of OGD, we know that 
	
	\begin{equation*}
		\begin{aligned}
			||x_{t+1} - x^*||^2 & = || \Pi_{\K} (x_t - \eta_t\nabla f_t(x_t))  - x^*||^2 \\
			& \leq || x_t - \eta_t\nabla f_t(x_t)  - x^*||^2  \\
			& = || x_t - x^*||^2 + || \eta_t\nabla f_t(x_t)||^2 - 2(\eta_t\nabla f_t(x_t))^T( x_t -  x^*).
		\end{aligned}
	\end{equation*}
	
	Therefore, we can get that 
	
	\begin{equation*}
		\begin{aligned}
				f_t(x_t) - f_t(x^*) & \leq \nabla_t^T (x_t - x^*) \\
									& \leq \frac{1}{2\eta_t} ( || x_t - x^*||^2 -  ||x_{t+1} - x^*||^2 ) + \frac{\eta_t}{2}| \nabla f_t(x_t)||^2 \\
									& \leq \frac{1}{2\eta_t} ( || x_t - x^*||^2 -  ||x_{t+1} - x^*||^2 ) + \frac{\eta_t G^2}{2}
		\end{aligned}
	\end{equation*}
	
	Define $\frac{1}{\eta_0} = 0$, and by summing all terms we can have
	
	\begin{equation*}
		\begin{aligned}
			Reg_T & = \sum\limits_{i=1}^T (f_t(x_t) - f_t(x^*))\\
					& \leq \sum\limits_{i=1}^T ( \frac{1}{2\eta_t} ( || x_t - x^*||^2 -  ||x_{t+1} - x^*||^2 ) + \frac{\eta_t G^2}{2}) \\
					& = \frac{1}{2} \sum\limits_{i=1}^T || x_t - x^*||^2 (\frac{1}{\eta_t} -\frac{1}{\eta_{t-1}})  +  \sum\limits_{i=1}^T  \frac{\eta_t G^2}{2} \\
					& =  \frac{|| x_t - x^*||^2}{2} \sum\limits_{i=1}^T  (\frac{1}{\eta_t} -\frac{1}{\eta_{t-1}})  + \frac{ G^2}{2}  \sum\limits_{i=1}^T  \eta_t \\
					& = \frac{|| x_t - x^*||^2}{2} \frac{1}{\eta_T}  + \frac{ G^2}{2}  \sum\limits_{i=1}^T  \eta_t
		\end{aligned}
	\end{equation*}
	
	By letting $\eta_t = \frac{D}{G\sqrt{t}}$, we can have that
	
	\begin{equation*}
		\begin{aligned}
				Reg_T & \leq  \frac{|| x_t - x^*||^2}{2} \frac{1}{ \frac{D}{G\sqrt{T}}}  + \frac{ G^2}{2}  \sum\limits_{i=1}^T  \frac{D}{G\sqrt{t}} \\
				& \leq \frac{D G\sqrt{T}}{2}   + \frac{ GD}{2}  \sum\limits_{i=1}^T  \frac{1}{\sqrt{t}}\\
				& \leq \frac{D G\sqrt{T}}{2}   + GD  \sqrt{T} \\
				& = \frac{3}{2}  GD  \sqrt{T}.
		\end{aligned} 
	\end{equation*}
\end{proof}

\begin{corollary}
	Gradient Descent algorithm for minimization one convex function $f(\cdot)$ with respect to averaging converge at a rate of $O(\frac{GD}{\sqrt{T}})$.
\end{corollary}

It's surprise to find that the convergence rate of OGD is the same as normal gradient descent algorithm when they output the averaging.

Next, we will talk about Stochastic Gradient Descent, its setting and convergence rate. We will see that we can use OGD to prove its convergence rate easily.

In stochastic gradient descent, basically we want to minimize $\arg\min\limits_{x\in \K} f(x)$, where $f(x) = \EE_{\xi}[h(x;\xi)]$. The basic procedures in SGD is following:

\begin{itemize}
	\item Sample $\xi_t$ from data set.
	\item $x_{t+1} \leftarrow x_t - \eta_t \nabla h(x_t; \xi_t)$.
	\item Output $\bar{x_t} = \frac{1}{T} \sum\limits_{i=1}^T x_t$.
\end{itemize}

We can have the following claim about SDG:

\begin{claim}
	Stochastic Gradient Descent converges at $O(\frac{DG}{\sqrt{T}})$.
\end{claim}

\begin{proof}
	Notice that
	
	\begin{equation*}
		\begin{aligned}
			\EE_{\xi_{1:t}} [f(\bar{x_T})] &\leq \frac{1}{T} \EE_{\xi_{1:t}} [ \sum\limits_{t=1}^T f(x_t)] \\
			&  = \frac{1}{T} \EE_{\xi_{1:t}} [ \sum\limits_{t=1}^T  \EE[f(x_t)|\xi_1, \cdots , \xi_{t-1} ]]  \\ 
			&  = \frac{1}{T} \EE_{\xi_{1:t}} [ \sum\limits_{t=1}^T  \EE[ \EE_{\xi}[h(x_t;\xi_{1:t})]|\xi_1, \cdots , \xi_{t-1} ]]  \\
			&  = \frac{1}{T} \EE_{\xi_{1:t}} [ \sum\limits_{t=1}^T  \EE[ \EE_{\xi}[h(x_t;\xi_{t})]|\xi_1, \cdots , \xi_{t-1} ]]  \\
			&  = \frac{1}{T} \EE_{\xi_{1:t}} [ \sum\limits_{t=1}^T  \EE[ h(x_t;\xi_{t})|\xi_1, \cdots , \xi_{t-1} ]]  \\
		\end{aligned}
	\end{equation*}
	
	Let $f_t(x) = h(x_t;\xi_t)$, then 
	
	\begin{equation*}
	\begin{aligned}
	\EE_{\xi_{1:t}} [f(\bar{x_T})] & \leq \frac{1}{T} \EE_{\xi_{1:t}} [ \sum\limits_{t=1}^T  \EE[ f_t(x_t)|\xi_1, \cdots , \xi_{t-1} ]]  \\ 
	&= \EE_{\xi_{1:t}} [\frac{1}{T} \sum\limits_{t=1}^T  f_t(x_t)] \\
	& \leq \EE_{\xi_{1:t}} [\frac{1}{T} \sum\limits_{t=1}^T  f_t(x^*)] + \frac{Regret_T(OGD)}{T} \\
	& = f(x^*) + \frac{3DG}{2\sqrt{T}}.
	\end{aligned}
	\end{equation*}
\end{proof}

Therefore, the convergence rate $\EE_{\xi_{1:t}} [f(\bar{x_T})] - f(x^*) \leq \frac{3DG}{2\sqrt{T}}$.


We now conclude the three settings we talked about.

\begin{itemize}
	\item $D$ - some distribution on data.
	\item $f(x, \xi)$ - a loss function of $x$ given data $\xi$.
	\item $F(x)$ - $\EE_{\xi \in D} [f(x;\xi)]$.
\end{itemize}

Imagine now we have a sample $\hat{D} = \{\xi_{1:n}\}$ and the empirical loss function is defined as $\hat{F}(x) = \frac{1}{n} \sum\limits_{i=1}^n f(x;\xi_i)$.

If the goal of learning is 

\begin{equation*}
	\textmd{minimize} \ \hat{F}(x),
\end{equation*} 

then its stochastic gradient descent.

If the goal of optimization is 

\begin{equation*}
\textmd{minimize} \ F(x),
\end{equation*} 

then its online to batch minimization.


Their procedure is almost the same:

\begin{algorithm}
	\floatname{algorithm}{}
	\begin{algorithmic}
		\STATE For $t = 1,\cdots ,T$:
		\bindent 
		\STATE Sample $\xi_t \sim D$ (or $\hat{D}$)
		\STATE $x_{t+1} = x_t - \eta_t\nabla f(x_t; \xi_t)$.
		\eindent
	\end{algorithmic}
\end{algorithm}


We have the following two claims for our algorithm.

\begin{claim}\label{claim_online_batch}
	Let $\bar{x_T} = \frac{1}{T} \sum\limits_{t=1}^T x_t$, then we have
	
	\begin{equation}
		\EE_{\xi_{1:n}} [F(\bar{x_T}) - F(x^*)] \leq O(\frac{GD}{\sqrt{T}}).
 	\end{equation}
\end{claim}

\begin{corollary}
	It is possible to get a high probability bound.
\end{corollary}

\begin{claim}\label{claim_sgd}
	Let $\bar{x_T} = \frac{1}{T} \sum\limits_{t=1}^T x_t$, then we have
	
	\begin{equation}
		\EE_{\xi_{1:n}} [\hat{F}(\bar{x_T}) - \hat{F}(x^*)] \leq O(\frac{GD}{\sqrt{T}}).
	\end{equation}
\end{claim}

We have proven claim \ref{claim_sgd} before, therefore, we will only prove claim \ref{claim_online_batch} here.

\begin{proof}
	We can see that 
	
	\begin{equation*}
		\begin{aligned}
			\EE_{\xi_{1:n}} [F(\bar{x_T}) - F(x^*)] &\leq \EE_{\xi_{1:n}} [\frac{1}{T}\sum\limits_{t=1}^T F(x_t) - F(x^*)] \\
			&= \EE_{\xi_{1:n}} [\frac{1}{T}\sum\limits_{t=1}^T \EE_{\xi \sim D} [f(x_t;\xi) - f(x^*;\xi)]] \\
			&= \EE_{\xi_{1:n}} [\frac{1}{T}\sum\limits_{t=1}^T \EE_{\xi_t} [f(x_t;\xi_t) - f(x^*;\xi_t)]] \\ &= \EE_{\xi_{1:n}} [\frac{1}{T}\sum\limits_{t=1}^T f(x_t;\xi_t) - f(x^*;\xi_t)] \\
			& = \EE [\frac{Regret_T}{T}] \\
			& \leq \frac{3GD}{2\sqrt{T}}.
		\end{aligned}
	\end{equation*}
\end{proof}

Note that the proof is not algorithm specific. That is, any low-regret algorithm can help us prove the theorem. The proof can guarantee the convergence of order $\EE[\frac{Reg_T}{T}]$ for any algorithm, and $\EE[\frac{Reg_T}{T}]$ will be small for a low-regret algorithm.

Another thing we need to consider is that we have shown that standard gradient descent converges at $O(\frac{GD}{\sqrt{T}})$, which is the same as SGD. Why is SGD better then?

The answer is that in each iteration, standard gradient descent will take $O(n)$ time while SGD will only use constant time. The time complexity in each iteration is the reason why SGD overtakes standard gradient descent.

Next, we will talk about a new online convex optimization algorithm, called Mirror Descent. The setting is following:

\begin{itemize}
	\item Given $\K \subset \R^d$ and a regularizer $R(\cdot)$.
	\item $\K \subset int(dom(R))$
	\item $R$ is $\lambda$-strongly convex (w.r.t $||\cdot||$).
	\item Let $\nabla_t = \nabla f_t(x_t)$ at time $t$.
\end{itemize}

Then the algorithm is 

\begin{algorithm}[H]
	\caption{Mirror Descent}
	\begin{algorithmic}
		\STATE $x_1 \in \K$ arbitrary.
		\STATE For $t = 1,\cdots ,T$:
		\bindent 
			\STATE $x_{t+1} = \arg\min\limits_{x\in \K} \eta_t \langle \nabla_t, x \rangle + D_R(x,x_t)$
		\eindent
	\end{algorithmic}
\end{algorithm}

where $D_R(x,x_t)$ is the Bregman Divergence 

\begin{equation*}
	D_R(x,y) = R(x) - R(y) - \langle \nabla R(y), x-y\rangle.
\end{equation*}

Since $R$ is $\lambda$-strongly convex, we know that $D_R(x,y) \geq \frac{\lambda}{2}||x-y||^2$.

Let's break down the update function of mirror descent. The term $\langle \nabla_t, x \rangle$ tends to move away from gradient. For example, if gradient $\nabla_t = [1, -1]$, then $x$ would be $[-\infty, \infty]$. To avoid infinity, we introduce $D_R(x,x_t)$, which will increase if $x$ moves too far away from $x_t$. Therefore, overall we can reach a balance of both parts.

Before proving its convergence, let's first introduce some lemmas.

\begin{lemma}\label{MD_lemma1}
	For any $x, y,z \in \K$, it holds that 
	\begin{equation}
		D_R(z,x) + D_R(x,y) - D_R(z,y) = \langle \nabla R(y) - \nabla R(x), z-x \rangle.
	\end{equation}
\end{lemma}

\begin{proof}
	By substituting $D_R(z,x)$ with the expression of Bregman Divergence, we can get the equality directly.
\end{proof}

\begin{lemma}[\bfseries First-order Optimal Condition]\label{MD_lemma2}
If $x^* = \arg\min\limits_{x \in \K} \Phi(x)$, then $\forall \ u\in \K$, we have

\begin{equation}
\langle \nabla \Phi(x^*), u -x^* \rangle \geq 0.
\end{equation}	
\end{lemma}

Let's denote $\Phi(x) = \eta_t \langle \nabla_t, x \rangle + D_R(x,x_t)$ and apply to mirror descent. In mirror descent, we have $ \nabla \Phi(x) = \eta_t \nabla_t + \nabla R(x) - \nabla R(x_t)$. Then by first-order optimal condition, we have that $\forall \ u \in \K$

\begin{equation*}
	\langle \eta_t \nabla_t + \nabla R(x_{t+1}) - \nabla R(x_t) , u - x_{t+1} \rangle \geq 0.
\end{equation*}

\begin{lemma}\label{MD_lemma3}
	
	Furthermore, notice that $\forall \ v,\gamma$, we have
	
	\begin{equation*}
	\langle v, \gamma \rangle \leq ||v|| \cdot ||\gamma||_* = (\frac{1}{\sqrt{\lambda}} ||v||)(\sqrt{\lambda} ||\gamma||_*) \leq \frac{||v||^2}{2\lambda} + \frac{||\gamma||_*^2\lambda}{2} .
	\end{equation*}
	
	The last inequality if Holder's inequality.
\end{lemma}

\begin{lemma}\label{MD_lemma4}
	The following inequality holds for any $u \in \K$,
	
	\begin{equation*}
	\eta_t (f_t(x_t) - f_t(u)) \leq \eta_t \langle \nabla_t, x_t -u \rangle \leq D_R(u,x_t) - D_R(u,x_{t+1}) + \frac{\eta_t^2}{2\lambda} ||\nabla_t||_*^2.
	\end{equation*}
\end{lemma}

\begin{proof}
	The first inequality $\eta_t (f_t(x_t) - f_t(u)) \leq \eta_t \langle \nabla_t, x_t -u \rangle $ is indicated by the convexity of $f$. Now let's look at the second inequality.
	\begin{equation*}
		\begin{aligned}
			\eta_t\langle \nabla_t, x_t -u \rangle =&  \langle \nabla R(x_t) - \nabla R(x_{t+1})-\eta_t \nabla_t , u -x_{t+1}\rangle  \\
													& -  \langle \nabla R(x_t) - \nabla R(x_{t+1})  , u -x_{t+1}\rangle  \\
													& +  \langle \eta_t \nabla_t  , x_t -x_{t+1}\rangle.
		\end{aligned}
	\end{equation*}


By Lemma \ref{MD_lemma2}, we know that $\langle \nabla R(x_t) - \nabla R(x_{t+1})-\eta_t \nabla_t , u -x_{t+1}\rangle \leq 0$. Therefore, we have

\begin{equation*}
	\begin{aligned}
	\eta_t\langle \nabla_t, x_t -u \rangle \leq  
	& -  \langle \nabla R(x_t) - \nabla R(x_{t+1})  , u -x_{t+1}\rangle  +  \langle \eta_t \nabla_t  , x_t -x_{t+1}\rangle \\
	\textmd{(Lemma \ref{MD_lemma1})} = & D_R(u,x_t)  - D_R(u,x_{t+1}) - D_R(x_{t+1},x_t)    +  \langle \eta_t \nabla_t  , x_t -x_{t+1}\rangle \\
	\textmd{(Lemma \ref{MD_lemma3})} \leq & D_R(u,x_t)  - D_R(u,x_{t+1}) - D_R(x_{t+1},x_t)    +   \frac{||\eta_t \nabla_t||_*^2}{2\lambda} + \frac{||x_t -x_{t+1}||^2\lambda}{2}\\
	(\lambda-\textmd{Convexity of } D) \leq & D_R(u,x_t)  - D_R(u,x_{t+1})   +   \frac{\eta_t^2|| \nabla_t||_*^2}{2\lambda}.
	\end{aligned}
\end{equation*}
\end{proof}
After all these lemma, we come to this theorem

\begin{theorem}
	For Mirror Descent, for any $u \in \K$, let $\eta_1 ,\cdots , \eta_t$ be decreasing, and let $d^2 = \max\limits_{t= 1,\cdots T} D_R(u,x_t)$, then we have
	 
	\begin{equation}
		\sum\limits_{t = 1}^T (f_t(x_t) - f_t(u)) \leq \frac{d^2}{\eta_T}+ \frac{1}{2\lambda} \sum\limits_{t=1}^T \eta_t ||\nabla_t||_*^2.
	\end{equation}
\end{theorem}

\begin{corollary}
	Let $\eta_t = \frac{d\sqrt{\lambda}}{\sqrt{\sum\limits_{s=1}^t ||\nabla_s||^2_*}}$, then
	\begin{equation*}
	Regret_T(MD) \leq 2 \frac{d}{\sqrt{\lambda}} \sqrt{\sum\limits_{t=1}^T ||\nabla_t||_*^2}.
	\end{equation*}
\end{corollary}

\begin{proof}
	By Lemma \ref{MD_lemma4}, we have that
	\begin{equation*}
		\begin{aligned}
			\sum\limits_{t = 1}^T (f_t(x_t) - f_t(u)) & \leq \sum\limits_{t = 1}^T \frac{1}{\eta_t}(D_R(u,x_t)  - D_R(u,x_{t+1})   +   \frac{\eta_t^2|| \nabla_t||_*^2}{2\lambda}) \\
			& = \frac{1}{\eta_1} D_R(u,x_1) - \frac{1}{\eta_T} D_R(u, x_{t+1})  + \sum\limits_{t = 1}^{T-1} (\frac{1}{\eta_{t+1}} - \frac{1}{\eta_t}) D_R(u,x_{t+1}) + \sum\limits_{t = 1}^T \frac{\eta_t^2|| \nabla_t||_*^2}{2\lambda} \\
			& \leq \frac{1}{\eta_1} d^2 + \sum\limits_{t = 1}^{T-1} (\frac{1}{\eta_{t+1}} - \frac{1}{\eta_t}) D_R(u,x_{t+1}) + \sum\limits_{t = 1}^T \frac{\eta_t^2|| \nabla_t||_*^2}{2\lambda}  \\
			& \leq \frac{1}{\eta_1} d^2 + d^2\sum\limits_{t = 1}^{T-1} (\frac{1}{\eta_{t+1}} - \frac{1}{\eta_t})  + \sum\limits_{t = 1}^T \frac{\eta_t^2|| \nabla_t||_*^2}{2\lambda}  \\
			& \leq \frac{d^2}{\eta_1}  + \frac{d^2}{\eta_T} - \frac{d^2}{\eta_1} + \sum\limits_{t = 1}^T \frac{\eta_t^2|| \nabla_t||_*^2}{2\lambda}  \\
			& = \frac{d^2}{\eta_T}  + \sum\limits_{t = 1}^T \frac{\eta_t^2|| \nabla_t||_*^2}{2\lambda}
		\end{aligned}
	\end{equation*}
\end{proof}

Online mirror descent can extend to many other online convex optimization algorithms, let's see two examples.

\begin{example}
	Let $\K$ be any convex set, and $R(x) = \frac{1}{2} ||x||^2$. By the update function of OMD, we can get that
	
	\begin{equation*}
		\begin{aligned}
		x_{t+1} &= \arg\min\limits_{x\in \K} \eta_t\langle x,\nabla_t \rangle + D_R(x,x_t) \\
		& = \Pi_{\K}(x_t - \eta \nabla_t)
		\end{aligned}		
	\end{equation*}
	
	which is online gradient descent.
\end{example}

\begin{example}
	Let $K = \Delta_n$ and $R(x) = \sum\limits_{i=1}^n x_i \log x_i$. Then
	
	\begin{equation*}
	x_{t+1} = \arg\min\limits_{x\in \K} \eta_t \langle x, \nabla_t \rangle + \sum\limits_{i=1}^n x_i \log \frac{x_i}{x_t(i)}.
	\end{equation*}
	
	By first order optimal condition, we can have that $(\eta_t \nabla_t)_i = \log\frac{x(i)}{x_t(i)}$, which means 
	
	\begin{equation*}
	x(i) = x_t(i) \exp(-\eta_t \nabla_t(i)) /Z.
	\end{equation*}
	
	It turns out that it is just exponential weights algorithm.
	
	Why do we choose this function $R$? Because entropy function is $1$-strongly convex with respect to $||\cdot||_1$. Then $d^2 = \max\limits_{t} D_R (u,x_t) \leq \log n$ and 
	
	\begin{equation*}
		Regret_T \leq \log n \sqrt{\sum\limits_{t=1}^T ||\nabla_t||^2_{\infty}}.
	\end{equation*}
\end{example}

Finally, we will talk about the last algorithm in online convex optimization: Follow the Regularized Leader algorithm.


\begin{algorithm}[H]
	\caption{Follow the Regularized Leader}
	\begin{algorithmic}
		\STATE $x_1 \in \K$ arbitrary.
		\STATE Regularizer $R(x)$
		\STATE For $t = 1,\cdots ,T$:
		\bindent 
		\STATE (V1): $x_{t+1} = \arg\min\limits_{x\in \K} \eta \sum\limits_{s=1}^t f_s(x) + R(x)$
		\STATE (V2): $x_{t+1} = \arg\min\limits_{x\in \K}  \eta \sum\limits_{s=1}^t \langle \nabla f_s (x_s), x \rangle + R(x)$
		\eindent
	\end{algorithmic}
\end{algorithm}

\begin{fact}$ $
	\begin{enumerate}
		\item If $f_s(x)$ is linear in $x$, then FTL can have linear regret.
		\item The second version of FTRL is just OMD al long as $x_t \in int(\K)$.
	\end{enumerate}
\end{fact}

Next we will prove the second fact.

\begin{proof}
	If $x_t \in int(\K)$, then for FTRL we have
	\begin{equation*}
		\begin{aligned}
			\nabla R(x_{t+1})  = - \eta \sum\limits_{s=1}^t \nabla f_s(x_s)  .
		\end{aligned}
	\end{equation*}

For OMD, we have

	\begin{equation*}
		\begin{aligned}
			\nabla R(x_{t+1}) &  = \nabla R(x_t) - \eta \nabla_t f_t(x_t) \\
							 & = \nabla R(x_{t-1}) - \eta (\nabla_t f_t(x_t) + \nabla_t f_{t-1}(x_{t-1})) \\
							 & \cdots \\
							 & = - \eta \sum\limits_{s=1}^t \nabla f_s(x_s) .
		\end{aligned}
	\end{equation*}
\end{proof}

Therefore, OMD is just FTRL when $x \in \K$. This also shows that the convergence rate of FTRL is as good as that of OMD.

This concludes our discussion about online convex optimization. Next, we will discuss Multi-Armed Bandit, which tries to balance exploration and exploitation.

\end{document}